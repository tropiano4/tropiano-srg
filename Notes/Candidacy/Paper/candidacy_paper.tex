%%%%%%%%%%%%%%%%%%%%%%%%%%%%%%%%%%%%%%%%%%%%%%%%%%%%%%%%%%%%%%%%%%%%%%%%%
%
% File: candidacy_paper.tex
%
% Author: A. J. Tropiano (tropiano.4@osu.edu)
% Date: July 1, 2019
%
% Candidacy paper on the status of nuclear optical potentials and future prospects.
%
% Revision history:
% 	07/02/19 --- Starting with bulleted outline.
%     07/10/19 --- Adding equations to formalism section.
%	07/11/19 --- Put theoretical issues section as part of the conclusion. Also, wrote much of the formalism section.
%
%%%%%%%%%%%%%%%%%%%%%%%%%%%%%%%%%%%%%%%%%%%%%%%%%%%%%%%%%%%%%%%%%%%%%%%%%


\documentclass[preprintnumbers,floatfix,aps,prc,preprint,nofootinbib]{revtex4-1}

% Packages
\usepackage{amsmath}
\usepackage{amsfonts}
\usepackage{amssymb}
\usepackage{bm}
\usepackage[font=small,skip=0pt]{caption} % For captions on figures and tables
\usepackage{cellspace}
\usepackage{color}
\usepackage{enumerate}
\usepackage{epsfig}
\usepackage[figuresright]{rotating}
\usepackage{float}
\usepackage{hyperref} % For clickable links to sections within table of contents
\usepackage{graphicx}
\graphicspath{{../../../Figures/Pictures/}} % Setting the graphics path
\usepackage{physics} % For bra-ket notation
\usepackage{siunitx}
\usepackage[caption=false]{subfig} % For sub-figures

\newcommand{\eps}{\varepsilon}


\begin{document}


%%%%%%%%%%%%%%%%%%%%%%%%%%%%%%%%%%%%%%%%%%%%%%%%%%%%%%%%%%%%%%%%%%%%%%%%%
\title{Status of nuclear optical potentials and future prospects}


\author{A.~J.~Tropiano}

\affiliation{\mbox{Department of Physics, The Ohio State University, Columbus, OH 43210, USA}}

\date{\today}

\maketitle

\newpage


%%%%%%%%%%%%%%%%%%%%%%%%%%%%%%%%%%%%%%%%%%%%%%%%%%%%%%%%%%%%%%%%%%%%%%%%%
\section{Introduction}
\label{sec:intro}


Things to include:
\\
-- Why are nuclear reactions important? (Processes that help us understand nuclear structure amongst other things. Exotic nuclei are short-lived and must used reactions to study them.) List examples.
\\
-- r-process for motivation.
\\
-- How do optical potentials help us understand nuclear reactions?
\\
-- References: ...


%%%%%%%%%%%%%%%%%%%%%%%%%%%%%%%%%%%%%%%%%%%%%%%%%%%%%%%%%%%%%%%%%%%%%%%%%
\section{Formalism}
\label{sec:formalism}


Things to include:
\\
-- Projectile strikes a nucleus: elastic scattering and more.
\\
-- Derivation of the general optical potential (equation (2.15) in Feshbach). \textcolor{red}{Check.}
\\
-- More general things from Thompson/Nunes: define optical potentials as complex potentials and consequences of this. Reaction cross section derivation?
\\
-- Drawbacks to this derivation. Transition to projection operator method.
\\
-- Relation to observables. Use Lippmann-Schwinger equation to bridge gap to observables: phase shifts and cross sections. Maybe this makes more sense in phenomenology section?
\\
-- Generalization: derivation with projection operators.
\\
-- References: \cite{Feshbach:1958nx}, \cite{Feshbach:1962ut}, \cite{thompson_nunes_2009}.
\\

\textcolor{red}{Add Feshbach reference somewhere.}
\\

First, we present the optical potential for an $A+1$ particle system consisting of an incident nucleon and a target nucleus of mass number $A$. The system is described by the Schr\"odinger equation
%
\begin{eqnarray}
	\label{eq:schrodinger_equation}
	\mathcal{H} \Psi = E \Psi,
\end{eqnarray}
%
with the Hamiltonian $\mathcal{H}$ given below:
%
\begin{eqnarray}
	\label{eq:total_hamiltonian}
	\mathcal{H} = H_A(\textbf{r}_1, \cdots , \textbf{r}_A) + T_0 + V(\textbf{r}_0, \cdots , \textbf{r}_A).
\end{eqnarray}
%
The variables $\textbf{r}_k$ correspond to position, spin, and isospin for the incident nucleon ($k=0$) and each nucleon in the target nucleus ($k=1 \cdots A$). $T_0$ is the kinetic energy of the incident nucleon and $V$ is the potential energy of the $A+1$ system. $H_A$ is the Hamiltonian for the target nucleus and satisfies the Schr\"odinger equation
%
\begin{eqnarray}
	\label{eq:nuclear_schrodinger_equation}
	H_A(\textbf{r}_1, \cdots , \textbf{r}_A) \psi_i(\textbf{r}_1, \cdots , \textbf{r}_A) = \epsilon_i \psi_i(\textbf{r}_1, \cdots , \textbf{r}_A),
\end{eqnarray}
%
for nuclear wave functions $\psi_i$ and energies $\epsilon_i$. Here, the index $i$ corresponds to each state of the target nucleus with $i=0$ being the ground state. The nuclear wave functions $\psi_i$ form a complete, orthonormal set; thus, we expand the wave function $\Psi$ as follows:
%
\begin{eqnarray}
	\label{eq:wave_function}
	\Psi(\textbf{r}_0, \cdots , \textbf{r}_A) = \sum_{i} \psi_i(\textbf{r}_1, \cdots , \textbf{r}_A) u_i(\textbf{r}_0).
\end{eqnarray}
%
Note, the factors $u_i$ carry the $\textbf{r}_0$ dependence.
\\

%
\textcolor{red}{Mention hard-core problem.} In the following, we suppress the coordinate, spin, and isospin dependencies for brevity. We substitute \ref{eq:wave_function} into the Schr\"odinger equation \ref{eq:schrodinger_equation} and use the orthonormality of $\psi_i$ to derive a system of equations for the amplitudes $u_i$:
%
\begin{eqnarray}
	\label{eq:u_equation}
	(T_0 + V_{ii} + \epsilon_i - E) u_i = - \sum_{j \neq i} V_{ij} u_j,
\end{eqnarray}
%
where the potential matrix elements are
%
\begin{eqnarray}
	\label{eq:potential_matrix_elements}
	V_{ij}(\textbf{r}_0) = \int{d^3 r_1 d^3 r_2 \cdots d^3 r_A \psi_i^* V \psi_j}.
\end{eqnarray}
%
Next, we would like to derive an uncoupled equation for $u_0$ to describe elastic scattering in which the target nucleus is in its ground state with an incident nucleon of energy $E$. The other indices $i$ describe an emergent nucleon in a different state (e.g. energy, spin, isospin, etc.) from the incident nucleon. It is convenient to define the vectors
%
\begin{eqnarray}
	\label{eq:u_vector}
	\Phi \equiv
	\begin{pmatrix}
		u_1 \\
		u_2 \\
		\vdots \\
	\end{pmatrix}
	,
\end{eqnarray}
%
\begin{eqnarray}
	\label{eq:potential_vector}
	\textbf{V}_0 =
	\begin{pmatrix}
		V_{01}, V_{02}, \cdots
	\end{pmatrix}
\end{eqnarray}
%
and the matrix operator $\textbf{H}$
%
\begin{eqnarray}
	\label{eq:hamiltonian_operator}
	H_{ij} = T_0 \delta_{ij} + V_{ij} + \epsilon_i \delta_{ij}.
\end{eqnarray}
%
Then we can rewrite \ref{eq:u_equation} as
%
\begin{subequations}
	\label{eq:u_vector_equation}
	\begin{eqnarray}
		\label{eq:u_vector_equation_a}
		(T_0 + V_{00} - E) u_0 = -\textbf{V}_0 \Phi,
	\end{eqnarray}
	\begin{eqnarray}
		\label{eq:u_vector_equation_b}
		(\textbf{H}-E) \Phi = -\textbf{V}_0^{\dagger} u_0.
	\end{eqnarray}
\end{subequations}
%
We solve \ref{eq:u_vector_equation_b} for $\Phi$
%
\begin{eqnarray}
	\label{eq:phi}
	\Phi = \frac{1}{E - \textbf{H} + i \eta} \textbf{V}_0^{\dagger} u_0,
\end{eqnarray}
%
where $\eta \rightarrow 0^+$ to ensure only outgoing waves are present in exit channels for $u_i$ with $i \geq 1$. Lastly, we substitute $\Phi$ into \ref{eq:u_vector_equation_a} to give
%
\begin{eqnarray}
	\label{eq:u0_equation}
	(T_0 + V_{00}  - \textbf{V}_0 \frac{1}{E-\textbf{H}+i\eta} \textbf{V}_0^{\dagger} - E) u_0 = 0,
\end{eqnarray}
%
and define the optical potential as
%
\begin{eqnarray}
	\label{eq:optical_potential}
	V_{opt} = V_{00}  - \textbf{V}_0 \frac{1}{E-\textbf{H}+i\eta} \textbf{V}_0^{\dagger}.
\end{eqnarray}
%
\textcolor{red}{Drawbacks of this derivation.}
\\

We can see from Eq.~\ref{eq:optical_potential} that the optical potential is complex and energy dependent. The factor of $i \eta$ leads to $V^{opt}$ being complex, and thus, non-hermitian. The factor of $i \eta$ accounts for incident particles leaving the entrance channel $u_0$ to an exit channel $u_i$ where $i \geq 1$. This only occurs if reactions are possible, that is, $E > \epsilon_1$. Because $V^{opt}$ is not hermitian, the S matrix is not unitary giving rise to complex scattering phase shifts.
\\

\textcolor{red}{Furthermore, the potential is non-local... Use reference \cite{Hodgson:1971ab}.}
\\

\textcolor{red}{%
-- Generalization by using projection operators.
\\
-- What does $P$ and $Q$ do?
\\
-- Examples? (Feshbach.)}
\\
The projection operators satisfy the following relations: $P+Q=1$, $P^2 = P$, and $Q^2 = Q$. We act on Eq.~\ref{eq:schrodinger_equation} with $P$ and $Q$ and use the projection operator relations to obtain two equations:
%
\begin{subequations}
	\label{eq:intermediate_effective_hamiltonian_equations}
	\begin{eqnarray}
		\label{eq:pp_pq}
		(E - \mathcal{H}_{PP}) P \Psi = \mathcal{H}_{PQ} Q \Psi,
	\end{eqnarray}
	\begin{eqnarray}
		\label{eq:qq_qp}
		(E - \mathcal{H}_{QQ}) Q \Psi = \mathcal{H}_{QP} P \Psi.
	\end{eqnarray}
\end{subequations}
%
Solving \ref{eq:qq_qp} for $Q \Psi$ yields
%
\begin{eqnarray}
	\label{eq:q_psi}
	Q \Psi = \frac{1}{E - \mathcal{H}_{QQ}} \mathcal{H}_{QP} P \Psi.
\end{eqnarray}
%
Note, if $P$ does not include all open channels, then a factor of $i \eta$ where $\eta \rightarrow 0^+$ must be inserted in the denominator as before to account for... \textcolor{red}{Finish this note.} Substituting $Q \Psi$ into Eq.~\ref{eq:pp_pq} and rearranging gives
%
\begin{eqnarray}
	\label{p_psi}
	(E - \mathcal{H}_{PP} - \mathcal{H}_{PQ} \frac{1}{E - \mathcal{H}_{QQ}} \mathcal{H}_{QP}) P \Psi = 0,
\end{eqnarray}
%
where the effective Hamiltonian is
%
\begin{eqnarray}
	\label{eq:effective_hamiltonian}
	H_{eff} = \mathcal{H}_{PP} + \mathcal{H}_{PQ} \frac{1}{E - \mathcal{H}_{QQ}} \mathcal{H}_{QP}.
\end{eqnarray}
%
\textcolor{red}{Advantages of the projection operator formulation.} Wider applicability with generalization of $P$.	


%%%%%%%%%%%%%%%%%%%%%%%%%%%%%%%%%%%%%%%%%%%%%%%%%%%%%%%%%%%%%%%%%%%%%%%%%
\section{Phenomenology}
\label{sec:phenomenology}


Things to include:
\\
-- Form of the potential: Woods-Saxon shape, coulomb component, spin-orbit force. (Basic example in Thompson/Nunes.). Can start with discussion similar to Thompson/Nunes.
\\
-- Types: best fit, local, and global (see Koning). Advantages and disadvantages of each.
\\
-- Fit strength, radii, and diffuseness of complex potential.
\\
-- Surface and volume component from Dickhoff?
\\
-- Issue: fitting ambiguities, extractions to exotic regions of the nuclear chart. Some short-comings mentioned in Koning.
\\
-- Make sure to touch on phenomenology of optical potentials in modern experimental analyses (key word is modern!) Koning.
\\
-- References: \cite{Dickhoff:2018wdd} section 3, \cite{Koning:2003zz}.



%%%%%%%%%%%%%%%%%%%%%%%%%%%%%%%%%%%%%%%%%%%%%%%%%%%%%%%%%%%%%%%%%%%%%%%%%
\section{Microscopic optical potentials}
\label{sec:microscopic}


Things to include:
\\
-- Successes and limitations.
\\
-- Motivation: predictions for exotic region of the nuclear chart.
\\
-- Major methods: Multiple scattering (see references in Dickhoff paper) and Green's function based methods (coupled cluster). SRG evolution.
\\
-- Coupled cluster Green's function \cite{Rotureau:2016jpf}.
\\
-- References: \cite{Dickhoff:2018wdd} section 4, \cite{Furumoto:2019anr} G-matrix interaction, \cite{Idini:2019hkq} self-consistent Green's function, \cite{Rotureau:2016jpf}.


%%%%%%%%%%%%%%%%%%%%%%%%%%%%%%%%%%%%%%%%%%%%%%%%%%%%%%%%%%%%%%%%%%%%%%%%%
\section{Theoretical issues}
\label{sec:section_4}


Things to include:
\\
-- Fitting ambiguities for phenomenological potential.
\\
-- Uncertainty quantification.
\\
-- Add this to outlook in conclusion?
\\
-- References: \cite{King:2018vzw}.


%%%%%%%%%%%%%%%%%%%%%%%%%%%%%%%%%%%%%%%%%%%%%%%%%%%%%%%%%%%%%%%%%%%%%%%%%
\section{Conclusion}
\label{sec:conclusion}


Summary and outlook.


%%%%%%%%%%%%%%%%%%%%%%%%%%%%%%%%%%%%%%%%%%%%%%%%%%%%%%%%%%%%%%%%%%%%%%%%%


\bibliography{../../tropiano_bib}


\end{document}