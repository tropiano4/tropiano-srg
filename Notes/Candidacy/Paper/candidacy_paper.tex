%%%%%%%%%%%%%%%%%%%%%%%%%%%%%%%%%%%%%%%%%%%%%%%%%%%%%%%%%%%%%%%%%%%%%%%%%
%
% File: candidacy_paper.tex
%
% Author: A. J. Tropiano (tropiano.4@osu.edu)
% Date: July 1, 2019
%
% Candidacy paper on the status of nuclear optical potentials and future prospects.
%
% Revision history:
% 	07/02/19 --- Starting with bulleted outline. 
%
%%%%%%%%%%%%%%%%%%%%%%%%%%%%%%%%%%%%%%%%%%%%%%%%%%%%%%%%%%%%%%%%%%%%%%%%%


\documentclass[preprintnumbers,floatfix,aps,prc,preprint,nofootinbib]{revtex4-1}

% Packages
\usepackage{amsmath}
\usepackage{amsfonts}
\usepackage{amssymb}
\usepackage{bm}
\usepackage[font=small,skip=0pt]{caption} % For captions on figures and tables
\usepackage{cellspace}
\usepackage{color}
\usepackage{enumerate}
\usepackage{epsfig}
\usepackage[figuresright]{rotating}
\usepackage{float}
\usepackage{hyperref} % For clickable links to sections within table of contents
\usepackage{graphicx}
\graphicspath{{../../../Figures/Pictures/}} % Setting the graphics path
\usepackage{physics} % For bra-ket notation
\usepackage{siunitx}
\usepackage[caption=false]{subfig} % For sub-figures

\newcommand{\eps}{\varepsilon}


\begin{document}


%%%%%%%%%%%%%%%%%%%%%%%%%%%%%%%%%%%%%%%%%%%%%%%%%%%%%%%%%%%%%%%%%%%%%%%%%
\title{Status of nuclear optical potentials and future prospects}


\author{A.~J.~Tropiano}

\affiliation{\mbox{Department of Physics, The Ohio State University, Columbus, OH 43210, USA}}

\date{\today}

\maketitle

\newpage


%%%%%%%%%%%%%%%%%%%%%%%%%%%%%%%%%%%%%%%%%%%%%%%%%%%%%%%%%%%%%%%%%%%%%%%%%
\section{Introduction}
\label{sec:intro}


Things to include:
\\
-- Why are nuclear reactions important? (Processes that help us understand nuclear structure amongst other things.) List examples.
\\
-- r-process for motivation.
\\
-- How do optical potentials help us understand nuclear reactions?
\\
-- References: ...


%%%%%%%%%%%%%%%%%%%%%%%%%%%%%%%%%%%%%%%%%%%%%%%%%%%%%%%%%%%%%%%%%%%%%%%%%
\section{Formalism}
\label{sec:formalism}


Things to include:
\\
-- More general things from Thompson/Nunes: define optical potentials as complex potentials and consequences of this. Reaction cross section derivation.
\\
-- Derivation of the general optical potential (equation (2.15) in Feshbach).
\\
-- Properties of the potential: complex (meaning), non-local (derivation), dispersion relation.
\\
-- Relation to observables (basic quantum stuff?)
\\
-- References: \cite{Feshbach:1958nx}, \cite{Feshbach:1962ut}, Thompson/Nunes textbook (73, 126).


%%%%%%%%%%%%%%%%%%%%%%%%%%%%%%%%%%%%%%%%%%%%%%%%%%%%%%%%%%%%%%%%%%%%%%%%%
\section{Phenomenology}
\label{sec:phenomenology}


Things to include:
\\
-- Form of the potential: Woods-Saxon shape, coulomb component, spin-orbit force. (Basic example in Thompson/Nunes.). Can start with discussion similar to Thompson/Nunes.
\\
-- Fit strength, radii, and diffuseness of complex potential.
\\
-- Issue: fitting ambiguities, extractions to exotic regions of the nuclear chart.
\\
-- Make sure to touch on phenomenology of optical potentials in modern experimental analyses (key word is modern!)
\\
-- References: \cite{Dickhoff:2018wdd} section 3, \cite{Koning:2003zz}.



%%%%%%%%%%%%%%%%%%%%%%%%%%%%%%%%%%%%%%%%%%%%%%%%%%%%%%%%%%%%%%%%%%%%%%%%%
\section{Microscopic optical potentials}
\label{sec:microscopic}


Things to include:
\\
-- Successes and limitations.
\\
-- Motivation: predictions for exotic region of the nuclear chart.
\\
-- Coupled cluster Green's function \cite{Rotureau:2016jpf}.
\\
-- References: \cite{Dickhoff:2018wdd} section 4, \cite{Furumoto:2019anr} G-matrix interaction, \cite{Idini:2019hkq} self-consistent Green's function, \cite{Rotureau:2016jpf}.


%%%%%%%%%%%%%%%%%%%%%%%%%%%%%%%%%%%%%%%%%%%%%%%%%%%%%%%%%%%%%%%%%%%%%%%%%
\section{Theoretical issues}
\label{sec:section_4}


Things to include:
\\
-- Fitting ambiguities for phenomenological potential.
\\
-- Uncertainty quantification.
\\
-- Add this to outlook in conclusion?
\\
-- References: \cite{King:2018vzw}.


%%%%%%%%%%%%%%%%%%%%%%%%%%%%%%%%%%%%%%%%%%%%%%%%%%%%%%%%%%%%%%%%%%%%%%%%%
\section{Conclusion}
\label{sec:conclusion}


Summary and outlook.


%%%%%%%%%%%%%%%%%%%%%%%%%%%%%%%%%%%%%%%%%%%%%%%%%%%%%%%%%%%%%%%%%%%%%%%%%


\bibliography{../../tropiano_bib}


\end{document}