%%%%%%%%%%%%%%%%%%%%%%%%%%%%%%%%%%%%%%%%%%%%%%%%%%%%%%%%%%%%%%%%%%%%%%%%%
%
% File: operator_evolution.tex
%
% Author: A. J. Tropiano (tropiano.4@osu.edu)
% Date: August 23, 2019
%
% Draft of paper on SRG operator evolution and the Magnus expansion.
%
% Revision history:
%	08/28/19 --- Started outlining and added pieces from old notes.
%	09/09/19 --- Added more bullets to the introduction. Further organization.
%	09/19/19 --- Added more bullets to section 2.
%
%%%%%%%%%%%%%%%%%%%%%%%%%%%%%%%%%%%%%%%%%%%%%%%%%%%%%%%%%%%%%%%%%%%%%%%%%


\documentclass[preprintnumbers,floatfix,aps,prc,preprint,nofootinbib]{revtex4-1}

% Packages
\usepackage{amsmath}
\usepackage{amsfonts}
\usepackage{amssymb}
\usepackage{bm}
\usepackage[font=small,skip=0pt]{caption} % For captions on figures and tables
\usepackage{cellspace}
\usepackage{color}
\usepackage{enumerate}
\usepackage{epsfig}
\usepackage[figuresright]{rotating}
\usepackage{float}
\usepackage{hyperref} % For clickable links to sections within table of contents
\usepackage{graphicx}
\graphicspath{{../../Figures/}} % Setting the graphics path
\usepackage{physics} % For bra-ket notation
\usepackage{siunitx}
\usepackage[caption=false]{subfig} % For sub-figures

\newcommand{\eps}{\varepsilon}


\begin{document}


%%%%%%%%%%%%%%%%%%%%%%%%%%%%%%%%%%%%%%%%%%%%%%%%%%%%%%%%%%%%%%%%%%%%%%%%%
\title{Operator evolution from the similarity renormalization group and the Magnus expansion}


\author{A.~J.~Tropiano$^{1}$, S.~K.~Bogner$^{2}$, R.~J.~Furnstahl$^{1}$}

\affiliation{%
		$^1$\mbox{Department of Physics, The Ohio State University, Columbus, OH 43210, USA}  \\
		$^2$\mbox{National Superconducting Cyclotron Laboratory and Department of Physics and Astronomy,}  \\
    		\mbox{Michigan State University, East Lansing, MI 48824, USA}
}

\date{\today}

\begin{abstract}

\noindent{Ideas for Magnus / SRG operator evolution paper}
\\
-- SRG/Magnus evolution in different potentials (non-local, local, semi-local). Universality. High cutoffs.
\\
-- Block-diagonal generator for high cutoff potentials and operator evolution. How the block-diagonal generator handles spurious bound states.
\\
-- Testing the Magnus expansion for high cutoff potentials using the potentials from Wendt 2011 for comparison. Spurious bound states and connection to intruder states in IMSRG calculations.
\\
-- Operator evolution for different potentials and generators.

\end{abstract}

\maketitle

\newpage


%%%%%%%%%%%%%%%%%%%%%%%%%%%%%%%%%%%%%%%%%%%%%%%%%%%%%%%%%%%%%%%%%%%%%%%%%
\section{Introduction}
\label{sec:intro}


\noindent{%
\textbf{Background on modern nuclear potentials.}
}
\\
-- Wide range of NN potentials.
%
% Add what you have from IMSRG paper.
\begin{itemize}
	\item Chiral EFT background.
	\item Different potentials but give same $S$-matrix.
\end{itemize}
%
-- Implementation to many-body calculations and SRG decoupling.
%
\begin{itemize}
	\item Strong coupling between low- and high-momentum matrix elements in NN potentials.
	\item Very difficult to implement these interactions in many-body methods using basis expansions. Matrix dimension becomes too large for accurate calculations.
	\item RG transformations are used to soften the interaction to make many-body methods feasible. One such method, the SRG, also preserves observables from unitarity.
	\item How do different potentials change under SRG transformations?
\end{itemize}
%
\textbf{SRG formalism}
\\
-- The SRG decouples low- and high-momentum scales by applying a continuous unitary transformation $U(s)$ where $s=0 \rightarrow \infty$ is the flow parameter.
\\
-- The `dressed' or evolved operator is given by
%
\begin{eqnarray}
	\label{eq:srg_operator}
	O(s) = U(s) O(0) U^{\dagger}(s),
\end{eqnarray}
%
where $O(0)$ corresponds to the `bare' operator.
\\
-- Because $U(s)$ is unitary, the observables of the operator are preserved.
\\
-- In practice, the unitary transformation U(s) is not explicitly solved for; the evolved operator is given by a differential flow equation which is obtained by taking the derivative of Eqn. (\ref{eq:srg_operator}),
%
\begin{eqnarray}
	\label{eq:srg_flow}
	\frac{dO(s)}{ds} = [\eta(s), O(s)],
\end{eqnarray}
%
where $\eta(s)=\frac{dU(s)}{ds} U^{\dagger}(s) = -\eta^{\dagger}(s)$ is the anti-hermitian SRG generator.
\\
-- The generator is defined as a commutator, $\eta(s) = [G, H(s)]$, where $G$ specifies the type of flow or form of the decoupled operator.
\\
-- By setting $G=H_D(s)$, the diagonal of the Hamiltonian, the operator is driven to band-diagonal form.
\\
-- This choice was implemented by Wegner in condensed matter physics \cite{Wegner:1994ab}.
\\
-- In a similar option used in nuclear physics, $G$ is set to the relative kinetic energy, $T_{rel}$, which also drives to band-diagonal form.
% Do I need to explain the intermediate step here?
\\
-- It is convenient to define $\lambda \equiv s^{-1/4}$ which roughly measures the width of the band-diagonal in the decoupled operator.
\\
-- For block-diagonal decoupling, denoted $G=H_{BD}(s)$, the operator is split into low- and high-momentum sub-blocks by specifying a separation in momentum $\Lambda_{BD}$.
\\
-- These transformations are similar to $V_{low k}$ Lee-Suzuki transformations but keep the high-momentum matrix elements non-zero, although entirely decoupled from the low-momentum sub-block.
\\
-- Generally the flow equation (\ref{eq:srg_flow}) is solved up to some finite value of $s$ with a high-order ODE solver.
\\
-- For notational convenience, we write the generators without the $s$ dependence in the rest of the paper.
\\
\textbf{Universality}
\\
% Try to understand this better. Ask DF.
-- The explicit long-range physics should be the same. Decoupling low- and high-energy gives matching low-energy matrix elements. In the NN potential, this means softened NN interactions should have the same low-momentum matrix elements after sufficient decoupling.
\\
-- Add takeaways from Dainton: phase shift equivalence implies matrix element equivalence for $\lambda$ approaching the momenta of phase equivalence. Correct binding energy is critical or the lowest matrix elements will not match.
\\
-- Motivate long list of unaddressed questions: regulator, generator dependence, high cutoffs, the Magnus expansion.
%
\begin{itemize}
	\item Regulator. Functional dependence of regulator. Reasons for implementing each.
	\item Generators. Band- and block-diagonal transformations.
	\item High cutoffs. Cite Nogga, Wendt, and Tews papers.
	\item The Magnus expansion. High cutoffs and connection to IMSRG intruder state problem.
\end{itemize}
%
\textbf{Operator evolution}
\\
-- State how a potential and wave function changes: how does this affect other operators?
\\
-- Operator evolution for different potentials (regulators, chiral order, etc.)
\\
-- How operators evolve from band- and block-diagonal SRG transformations.
\\
\textbf{Overview of sections.}


%%%%%%%%%%%%%%%%%%%%%%%%%%%%%%%%%%%%%%%%%%%%%%%%%%%%%%%%%%%%%%%%%%%%%%%%%
\section{SRG evolution of NN potentials}
\label{sec:srg_evolution_nn_potentials}


\noindent{%
\textbf{General outline of the section}
}
\\
-- Comparison of potential evolution with different regulators, orders, generators.
\\
-- Universality.
\\
-- Discussion of high cutoffs, block-diagonal generator at high cutoffs, and how it handles spurious bound states.
\\
-- Use high cutoffs to transition to Magnus test problem.
\\
\textbf{Analysis of figures}
\\
-- Fig.~\ref{fig:potential_contours_RKE_Wegner} illustrates the SRG in a nutshell. Here, we evolve three partial wave channels of RKE N$^3$LO \cite{Reinert:2017usi} where the cutoff $\Lambda=450$ MeV to $\lambda=1.5$ fm$^{-1}$. We see that the off-diagonal elements of the potential approach zero and the potential is driven to band-diagonal form.
%
\begin{figure}[H]
	\centering
	\subfloat[]{%
	\includegraphics[clip,width=0.9\columnwidth]{SRG_potentials/potential_contours_kvnn106_1S0_Wegner_channel_label}%
	}
	
	\subfloat[]{%
	\includegraphics[clip,width=0.9\columnwidth]{SRG_potentials/potential_contours_kvnn106_3S1_Wegner_channel_label}%
	}

	\subfloat[]{%
	\includegraphics[clip,width=0.9\columnwidth]{SRG_potentials/potential_contours_kvnn106_1P1_Wegner_channel_label}%
	}
	\caption{Matrix elements of the RKE N$^3$LO potential SRG-evolving in $\lambda$ right to left under transformations with the Wegner generator in the $^1$S$_0$ (a), $^3$S$_1$ (b), and $^1$P$_1$ (c) channels.}
	\label{fig:potential_contours_RKE_Wegner}
\end{figure}
%
\noindent{%
-- In Fig.~\ref{fig:potential_contours_3S1_Wegner} we consider three different SRG-evolved potentials in the $^3$S$_1$ channel: EM N$^3$LO (500 MeV cutoff) \cite{Entem:2003ft}, RKE N$^3$LO (450 MeV cutoff) \cite{Reinert:2017usi}, and Gezerlis et al. N$^2$LO (1 fm cutoff) \cite{Gezerlis:2014zia}. The major difference in these three potentials are the regulator functions in the contact and pion-exchange terms. The EM N$^3$LO interaction is a non-local potential where both contact and pion-exchange interactions feature a non-local regulator function of the form exp$[-(p/\Lambda)^{2n}-(p'/\Lambda)^{2n}]$, where $\Lambda$ is the momentum-space cutoff and $n$ is an integer. However, a non-local regulator function for pion-exchange contributions can introduce regulator artifacts for cutoffs $\Lambda$ lower than the breakdown scale $\Lambda_b$. Several semi-local chiral potentials have been introduced to reduce regulator artifacts, such as the RKE N$^3$LO potential. Here, a local regulator function is applied for the long-range interactions in momentum-space, while a non-local regulator function is used for the short-range interactions. In some instances, non-local interactions are not suitable for $\textit{ab initio}$ approaches such as the quantum Monte Carlo (QMC) method motivating the need for fully local potentials. The Gezerlis et al. N$^2$LO potential is an example of a local interaction where the long-range terms have a local regulator function in coordinate-space and the short-range terms have a local regulator function in momentum-space.
}
\\
-- Takeaways from Fig.~\ref{fig:potential_contours_3S1_Wegner}: completely different at $\lambda=6$ fm$^{-1}$ but low-momentum matrix elements are similar at $\lambda=1$ fm$^{-1}$. Decoupled, low-momentum matrix elements are necessarily the same since the pion-exchange terms are calculated explicitly. (Cutoff dependence can play a role though for lower cutoffs.)
%
\begin{figure}[H]
	\centering
	\subfloat[]{%
	\includegraphics[clip,width=0.9\columnwidth]{SRG_potentials/potential_contours_kvnn10_3S1_Wegner_potential_label}%
	}
	
	\subfloat[]{%
	\includegraphics[clip,width=0.9\columnwidth]{SRG_potentials/potential_contours_kvnn106_3S1_Wegner_potential_label}%
	}

	\subfloat[]{%
	\includegraphics[clip,width=0.9\columnwidth]{SRG_potentials/potential_contours_kvnn222_3S1_Wegner_potential_label}%
	}
	\caption{Matrix elements of the EM N$^3$LO (a), RKE N$^3$LO (b), and Gezerlis et al. N$^2$LO (c) potentials SRG-evolving in $\lambda$ right to left under transformations with the Wegner generator in the $^3$S$_1$ channel.}
	\label{fig:potential_contours_3S1_Wegner}
\end{figure}
%
\noindent{%
-- Fig.~\ref{fig:potential_contours_3S1_RKE} shows the SRG-evolved RKE N$^3$LO (450 MeV cutoff) potential in the $^3$S$_1$ channel for two SRG generators: the Wegner and block-diagonal generators which drive the potential to band- and block-diagonal form, respectively. We continue to evolve to band-diagonal form with respect to the parameter $\lambda$, but for the block-diagonal generator, we label sub-plots with the parameter $\Lambda$ which characterizes the sharp cutoff in decoupling the low- and high-momentum matrix elements.
}
\\
-- Takeaways from Fig.~\ref{fig:potential_contours_3S1_RKE}: smooth decoupling for Wegner and sharp for block-diagonal, each unique generator should have its own type of universality. Check this more quantitatively by comparing matrix element ``slices''.
%
\begin{figure}[H]
	\centering
	\subfloat[]{%
	\includegraphics[clip,width=0.9\columnwidth]{SRG_potentials/potential_contours_kvnn106_3S1_Wegner_generator_label}%
	}
	
	\subfloat[]{%
	\includegraphics[clip,width=0.9\columnwidth]{SRG_potentials/potential_contours_kvnn106_3S1_Block-diag_generator_label}%
	}
	\caption{Matrix elements of the RKE N$^3$LO potential SRG-evolving right to left under transformations with Wegner (a) and block-diagonal (b) generators in the $^3$S$_1$ channel. Here, we use $\lambda$ for Wegner evolution in the top row and $\Lambda$ for block-diagonal evolution in the bottom row. For block-diagonal evolution, we fix $\lambda=1.5$ fm$^{-1}$.}
	\label{fig:potential_contours_3S1_RKE}
\end{figure}
%
\noindent{%
-- Fig.~\ref{fig:phase_shifts} shows the NN phase shifts of EM N$^3$LO, RKE N$^3$LO, and Gezerlis et al. N$^2$LO potentials in the $^1$S$_0$, $^3$S$_1$, and $^1$P$_1$ channels. In \cite{Dainton:2013axa}, it was found that phase equivalence up to some value of momentum $k$ implies matrix element equivalence up to the same value of $k$ in SRG-evolved potentials. We verify this conclusion by checking the matrix elements in Fig.~\ref{fig:potential_slices_1S0} where we see a collapse of the different potential matrix elements to the same line in the last column ($\lambda=1$ fm$^{-1}$). Note, Figs.~\ref{fig:potential_slices_1S0}-\ref{fig:potential_slices_1P1} show both Wegner and block-diagonal evolution where the solid lines correspond to the Wegner generator and dash-dotted to block-diagonal evolution. We see each generator collapses the potential to a different form because the low- and high-momentum matrix elements decouple in a different manner.
}
\\
-- In the $^3$S$_1$ channel, we see a slight deviation in the lowest momentum potential matrix element from EM N$^3$LO and the other two potentials. This is due to a minor difference in the deuteron binding energy ($\approx 1\%$ difference).
\\
-- The bottom row of Fig.~\ref{fig:potential_slices_1P1} shows $V(k,0.5)$ instead of $V(k,0)$ because the far off-diagonal matrix elements in the $^1$P$_1$ channel are all zero.
\begin{figure}[H]
	\centering
	\subfloat[]{%
	\includegraphics[clip,width=0.3\columnwidth]{SRG_observables/phase_shifts_1S0_kvnns_10_106_222}%
	}
	\quad
	\subfloat[]{%
	\includegraphics[clip,width=0.3\columnwidth]{SRG_observables/phase_shifts_3S1_kvnns_10_106_222}%
	}
	\quad
	\subfloat[]{%
	\includegraphics[clip,width=0.3\columnwidth]{SRG_observables/phase_shifts_1P1_kvnns_10_106_222}%
	}
	\caption{$^1$S$_0$ (a), $^3$S$_1$ (b), and $^1$P$_1$ (c) phase shifts for the EM N$^3$LO (solid black), RKE N$^3$LO (red dash-dotted), and Gezerlis et al. N$^2$LO (blue dashed) potentials.}
	\label{fig:phase_shifts}
\end{figure}
%
\begin{figure}[H]
	\centering
	\subfloat[]{%
	\includegraphics[clip,width=0.9\columnwidth]{SRG_potentials/potential_diag_1S0_kvnns_10_106_222_lamb1,0}%
	}
	
	\subfloat[]{%
	\includegraphics[clip,width=0.9\columnwidth]{SRG_potentials/potential_off-diag_1S0_kvnns_10_106_222_lamb1,0}%
	}
	\caption{Diagonal (a) and far off-diagonal (b) matrix elements of the EM N$^3$LO (black), RKE N$^3$LO (red) and Gezerlis et al. N$^2$LO (blue) potentials SRG-evolving right to left under transformations with Wegner (solid) and block-diagonal (dash-dotted) generators in the $^1$S$_0$ channel. Here, we use $\lambda$ for Wegner evolution and $\Lambda$ for block-diagonal evolution. For block-diagonal evolution, we fix $\lambda=1$ fm$^{-1}$.}
	\label{fig:potential_slices_1S0}
\end{figure}
%
\begin{figure}[H]
	\centering
	\subfloat[]{%
	\includegraphics[clip,width=0.9\columnwidth]{SRG_potentials/potential_diag_3S1_kvnns_10_106_222_lamb1,0}%
	}
	
	\subfloat[]{%
	\includegraphics[clip,width=0.9\columnwidth]{SRG_potentials/potential_off-diag_3S1_kvnns_10_106_222_lamb1,0}%
	}
	\caption{Same as Fig.~\ref{fig:potential_slices_1S0} but in the $^3$S$_1$ channel.}
	\label{fig:potential_slices_3S1}
\end{figure}
%
\begin{figure}[H]
	\centering
	\subfloat[]{%
	\includegraphics[clip,width=0.9\columnwidth]{SRG_potentials/potential_diag_1P1_kvnns_10_106_222_lamb1,0}%
	}
	
	\subfloat[]{%
	\includegraphics[clip,width=0.9\columnwidth]{SRG_potentials/potential_off-diag_1P1_kvnns_10_106_222_lamb1,0}%
	}
	\caption{Same as Figs.~\ref{fig:potential_slices_1S0} and \ref{fig:potential_slices_3S1} but in the $^1$P$_1$ channel.}
	\label{fig:potential_slices_1P1}
\end{figure}
%
\noindent{%
\textbf{High cutoffs}
}
\begin{itemize}
	\item Cutoff dependence in non-local LO (Wendt) potential.
	\item Band- and block-diagonal evolution and universality. Add figure analogous to Fig.~\ref{fig:potential_slices_3S1} but for $\Lambda=4$, $9$, and $20$ fm$^{-1}$.
	\item Spurious bound state(s).
	\item Discussion on how the block-diagonal generator handles spurious bound states. Where they are ``decoupled'' in the matrix compared to Wegner.
	\item Transition to Magnus expansion.
\end{itemize}
%
\begin{figure}[H]
	\centering
	\subfloat[]{%
	\includegraphics[clip,width=0.9\columnwidth]{SRG_potentials/potential_diag_3S1_kvnns_900_901_902_lamb1,2}%
	}
	
	\subfloat[]{%
	\includegraphics[clip,width=0.9\columnwidth]{SRG_potentials/potential_off-diag_3S1_kvnns_900_901_902_lamb1,2}%
	}
	\caption{Diagonal (a) and far off-diagonal (b) matrix elements of the non-local LO potentials at cutoffs $\Lambda=4$ (black), $9$ (red) and $20$ (blue) fm$^{-1}$ SRG-evolving right to left under transformations with Wegner (solid) and block-diagonal (dash-dotted) generators in the $^3$S$_1$ channel. Here, we use $\lambda$ for Wegner evolution and $\Lambda$ for block-diagonal evolution. For block-diagonal evolution, we fix $\lambda=1.2$ fm$^{-1}$.}
	\label{fig:potential_slices_high_cutoffs}
\end{figure}
%


%%%%%%%%%%%%%%%%%%%%%%%%%%%%%%%%%%%%%%%%%%%%%%%%%%%%%%%%%%%%%%%%%%%%%%%%%
\section{The Magnus expansion}
\label{sec:magnus_expansion}


\noindent{%
-- Connection to IMSRG intruder state.
}


% - - - - - - - - - - - - - - - - - - - - - - - - - - - - - - - - - - - - - - - - - - - - - - - - - - - - - - - - - - - - - - - - - - - - - - - - - - - - - - - - - - - - - - - - - - - - - - - - - - - - - - - - 
\subsection{Formalism}
\label{sec:magnus_expansion_formalism}


\noindent{%
-- Motivation: simplifies computational problem for evolving multiple operators, exact unitarity.
}
\\
-- We now consider the Magnus implementation.
\\
-- Mathematically speaking, the Magnus expansion is a method for solving an initial value problem associated with a linear ordinary differential equation (ODE).
\\
-- Formal details of the Magnus expansion are discussed in \cite{Blanes:2009ab}.
\\
-- We will introduce the Magnus expansion in the context of SRG evolving any operator.
\\
-- In an intermediate step in deriving Eqn. (\ref{eq:srg_flow}), we have a linear ODE for $U(s)$,
%
\begin{eqnarray}
	\label{eq:unitary_trans}
	\frac{dU(s)}{ds} = \eta(s) U(s).
\end{eqnarray}
%
-- Magnus showed that one can solve the following equation with a solution $U(s)=e^{\Omega(s)}$ where $\Omega(s)$ is expanded as a power series, $\sum_{n}^{\infty} \Omega_n$ (referred to as the Magnus expansion or Magnus series).
\\
-- The terms of the series are given by integral expressions involving $\eta(s)$ (again, see \cite{Blanes:2009ab, Magnus:1954zz} for details).
\\
-- For our case, we focus on the formally exact derivative of $\Omega(s)$,
%
\begin{eqnarray}
	\label{eq:magnus_omega}
	\frac{d\Omega(s)}{ds} = \sum_{k=0}^{\infty} \frac{B_k}{k!} ad_{\Omega}^{k}(\eta),
\end{eqnarray}
%
where $B_k$ are the Bernoulli numbers, $ad_{\Omega}^{0}(\eta)=\eta(s)$, and $ad_{\Omega}^{k}(\eta)=[\Omega(s),ad_{\Omega}^{k-1}(\eta)]$.
\\
-- We integrate this differential equation to find $\Omega(s)$ and evaluate the unitary transformation directly.
\\
-- Then the evolved operator can be evaluated with the BCH formula:
%
\begin{eqnarray}
	\label{eq:bch}
	O(s) = e^{\Omega(s)} O e^{-\Omega(s)} = \sum_{k=0}^{\infty} \frac{1}{k!} ad_{\Omega}^{k}(O).
\end{eqnarray}
%
-- As $k \rightarrow \infty$ in both sums in Eqns. (\ref{eq:magnus_omega}) and (\ref{eq:bch}) the Magnus transformation matches the SRG transformation exactly.
\\
-- We investigate several truncations $k_{max}$ in Eqn. (\ref{eq:magnus_omega}) and take many terms, $k_{max} \sim 25$, in Eqn. (\ref{eq:bch}).
\\
\textcolor{red}{%
-- Here or earlier (for the following bullets)? Better to motivate the Magnus in the introduction or easier to explain given mathematical detail?
}
\\
-- There are significant advantages in the Magnus implementation.
\\
-- In the typical approach, the numerical error associated with solving the flow equation affects the accuracy of the observables for the evolved operator.
\\
-- Therefore, one must use a high-order ODE solver in integrating the flow equation (\ref{eq:srg_flow}).
\\
-- In the Magnus implementation, unitarity is guaranteed by the form of $U(s)$; in fact, one could solve Eqn. (\ref{eq:magnus_omega}) with a simple first-order Euler step-method keeping the same observables while decoupling the operator as desired.
\\
-- This offers a decent computational speed-up by avoiding a high-order solver.
\\
-- In this paper, we demonstrate this advantage by applying the Magnus implementation using the first-order Euler step-method.
\\
-- The second major advantage involves the evolution of multiple operators.
\\
-- In many other situations, one may be interested in evolving several operators at a time.
\\
-- In the SRG procedure, we would have another set of coupled equations in Eqn. (\ref{eq:srg_flow}), drastically increasing memory usage.
\\
-- Each additional operator increases the set of equations - say $N$ equations - by another factor of $N$.
\\
-- In the Magnus, one only needs $\Omega(s)$ to consistently evolve several operators.
\\
-- We avoid the cost in memory by directly constructing $U(s)=e^{\Omega(s)}$.
\\
-- This is especially useful in IMSRG calculations where the model space can be very large.
\\
-- In the next section, we discuss results from Magnus-evolved large-cutoff potentials focusing on the flow of the potential, observables, and operator evolution.


% - - - - - - - - - - - - - - - - - - - - - - - - - - - - - - - - - - - - - - - - - - - - - - - - - - - - - - - - - - - - - - - - - - - - - - - - - - - - - - - - - - - - - - - - - - - - - - - - - - - - - - - - 
\subsection{Results}
\label{sec:magnus_expansion_results}


\noindent{%
-- Comparison to Wendt problem.
}
\\
-- Implications for IMSRG.
\\
-- Use discussion of operator evolution to transition to next section.
%
\begin{figure}[H]
	\centering
	\includegraphics[clip,width=0.9\columnwidth]{Magnus/potential_diagonals_offdiags_kvnn901_Wegner}%
	%\caption{Caption.}
	\label{fig:potential_slices_high_cutoffs_Wegner}
\end{figure}
%
\begin{figure}[H]
	\centering
	\includegraphics[clip,width=0.9\columnwidth]{Magnus/potential_diagonals_offdiags_kvnn901_T}%
	%\caption{Caption.}
	\label{fig:potential_slices_high_cutoffs_T}
\end{figure}
%


%%%%%%%%%%%%%%%%%%%%%%%%%%%%%%%%%%%%%%%%%%%%%%%%%%%%%%%%%%%%%%%%%%%%%%%%%
\section{Evolution of other operators}
\label{sec:evolution_other_operators}


\noindent{%
-- SRG operator evolution for different potentials and generators.
}
\\
-- General questions to address: universality for operators, different generators.
\\
-- Momentum projection operator.
%
\begin{itemize}
	\item Contours and general behavior: SRG transformation shifts strength of operator to low-momentum.
	\item Deuteron momentum distributions. Why does this make sense with the transformed operators?
	\item Diagonals and far off-diagonals. Universality.
	\item Same questions but with figures of the integrand.
\end{itemize}
%
-- $\hat{r}^2$ operator.


%%%%%%%%%%%%%%%%%%%%%%%%%%%%%%%%%%%%%%%%%%%%%%%%%%%%%%%%%%%%%%%%%%%%%%%%%
\section{Conclusion}
\label{sec:conclusion}


\noindent{%
-- Summary.
}
\\
-- Outlook.


%%%%%%%%%%%%%%%%%%%%%%%%%%%%%%%%%%%%%%%%%%%%%%%%%%%%%%%%%%%%%%%%%%%%%%%%%


\bibliography{../tropiano_bib}

\end{document}