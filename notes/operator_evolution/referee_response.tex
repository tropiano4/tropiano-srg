% Response to referee comments.

\documentclass[preprintnumbers,floatfix,aps,prc,preprint,nofootinbib]{revtex4-1}


\usepackage{physics} % For bra-ket notation

\newcommand{\ataq}{a^{\dagger}_q a_q}
\newcommand{\PP}{\ensuremath{P\mbox{--}P}}
\newcommand{\PQ}{\ensuremath{P\mbox{--}Q}}
\newcommand{\QP}{\ensuremath{Q\mbox{--}P}}
\newcommand{\QQ}{\ensuremath{Q\mbox{--}Q}}
\newcommand{\Klo}{K_{\text{lo}}}
\newcommand{\Khi}{K_{\text{hi}}}
\newcommand{\LambdaBD}{{\Lambda_{\text{BD}}}}


\begin{document}

We thank the referee for her/his careful reading of our paper and for the questions and suggestions. We address each of the points in turn.

\begin{enumerate}

    \item ``The results of section II nicely demonstrate the benefits of the Magnus expansion. I would find it useful if the authors could add some brief discussion on the computational cost of the Magnus expansion (as a function of $k_{\rm max}$) compared to the traditional SRG evolution for a given choice of tolerance for the ODE solver.''
    

    For the range of applications in the paper, the computational cost is roughly comparable to the conventional free-space SRG.
    While the evaluation of the RHS of the Magnus formulation Eq.~(7) is more complicated with increasing $k_{\rm max}$, it allows the use of a simple ODE solver such as the first-order Euler method, which takes fewer evaluations of the RHS than a typical adaptive high-order solver required to solve Eq.~(2) to comparable precision.
    We've added a sentence to this effect in section II B (on page 5).
    To reiterate what we say in the paper, the computational benefits of the Magnus approach are particularly for the in-medium SRG.
    
    
    \item ``At the beginning of section III the authors should add a statement to clarify to what extend the studies in this section are connected to the previous section. First, I assumed that the SRG studies would also be based on the Magnus expansion, this time applied to the high-order NN interaction. However, it seems that these two sections are not immediately related. Is there a reason why the Magnus method was not used for the calculations? Is the agreement between the Magnus method and the traditional evolution expected to be comparable to the results shown in section II?''

    
    We use the SRG for sections III-IV because the Magnus implementation and typical SRG approach are indistinguishable with soft potentials, unlike the case in section II where we tested high-cutoff potentials, which are intrinsically hard. We added a statement clarifying this at the end of the first paragraph in section III.
    
    \item ``I find the results shown if Fig.~16 somewhat confusing, especially those in the left three panels. What is the significance of the apparent oscillatory behavior? In the main text these features are attributed to the 'dependence on the mesh'. What does that mean? Is it then even useful to study these matrix elements? I would naively expect that the P-P part in the left panel should be a smooth function without the visible oscillatory artefacts. It would be helpful for me if the authors could comment on that in some more detail.''
    
    
    The oscillations seen in Fig.~16 are dependent on the bare $r^2$ operator, which depends on its regularization (which is what we meant by dependent on the mesh) as illustrated in Figs.~14 and 15.
    It is useful to visualize these results, which illustrate factorization and in particular the form of the \QQ\ part.
    We can see the origin of the oscillations by using factorization of the SRG transformation, where $U(k,q) \approx \Klo(k) \Khi(q)$ and from Ref.~[26], $U(k'',k') \approx Z(\LambdaBD) \delta(k''-k')$ where $k,k',k'' < \LambdaBD \ll q$.
    We can take the $\QP$ contribution to $Pr^2P$ as an example. It has the approximate form 
    \begin{align}
       \Klo(k) &\int_{\LambdaBD}^{\infty} dq\, \int_0^{\LambdaBD} dk''\, \Khi(q) \mel{q}{r^2}{k''} Z(\LambdaBD) \delta(k''-k')
       \notag \\
       & \approx 
       \Klo(k) \int_{\LambdaBD}^{\infty} dq\, \Khi(q) \mel{q}{r^2}{k'} Z(\LambdaBD)
    \end{align}
    The $k$-dependence drops out (to leading order $\Klo(k)$ is a constant), and the $k'$-dependence is oscillatory due to the form of the bare $r^2$.
    This explains the vertical stripes in the middle two panels of Fig.~16 being nearly constant within each stripe.
    The oscillatory artefacts are based on setting a sharp cutoff in coordinate space (i.e., an unregulated coordinate-space operator, see Fig.~14 for momentum-space matrix elements).
    Indeed, we can regulate the coordinate-space $r^2$ operator and smooth out the oscillations as in Fig. 15, which keeps a similar induced form in the $\QQ$ panel.
    The oscillations in the \PP\ panel are emphasized because we subtract out the initial $P r^2 P$ operator in Fig.~16;
    we have added a note about this in the caption of Fig.~16.
    At the end of section IV A we added some discussion on the non-$\QQ$ panels, pointing to section IV C for more detail.
    In section IV C, we added further explanation of the terms in Eq.~(15) and Fig.~16 using factorization.
    
    
    \item ``It seems that several of the presented results in section IV are directly related to the discussions and results presented in Ref.~[26]. It would be helpful if the authors could add a paragraph in which they summarize the new aspects and insights, as well as differences and similarities of the discussed results compared to Ref.~[26].''
    
    
    Our results in Sec.~IV heavily build on the results of Ref.~[26], but extend the conclusions of Ref.~[26] with an emphasis on scheme dependence by contrasting several NN potentials, utilizing band- and block-diagonal SRG decoupling, and varying the bare operators through the implementation of smeared delta functions in $\ataq$ or adding a coordinate-space regulator to $r^2$. We highlight the different forms of scheme dependence in several examples throughout the section. 
    We also apply factorization more widely (as for the $r^2$ analysis) and discuss the implications for different knock-out experiments.
    To summarize these new aspects, we added a paragraph to the end of Sec.~IV on how scheme dependence affects our conclusions on SRG operator evolution.


\end{enumerate}


\end{document}